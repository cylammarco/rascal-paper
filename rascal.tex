%                                                                 aa.dem
% AA vers. 9.1, LaTeX class for Astronomy & Astrophysics
% demonstration file
%                                                       (c) EDP Sciences
%-----------------------------------------------------------------------
%
%\documentclass[referee]{aa} % for a referee version
%\documentclass[onecolumn]{aa} % for a paper on 1 column  
%\documentclass[longauth]{aa} % for the long lists of affiliations 
%\documentclass[letter]{aa} % for the letters 
%\documentclass[bibyear]{aa} % if the references are not structured 
%                              according to the author-year natbib style

%
\documentclass{aa}  

%
\usepackage{graphicx}
%%%%%%%%%%%%%%%%%%%%%%%%%%%%%%%%%%%%%%%%
\usepackage{txfonts}
%%%%%%%%%%%%%%%%%%%%%%%%%%%%%%%%%%%%%%%%
%\usepackage[options]{hyperref}
% To add links in your PDF file, use the package "hyperref"
% with options according to your LaTeX or PDFLaTeX drivers.
%
\begin{document} 

   \title{Ransac Assisted Spectral CALibration~(RASCAL)}

   \subtitle{Towards automated spectral wavelength calibration}

   \author{J. Veitch-Michaelis
          \inst{1, 3}\fnmsep\thanks{veitchjo@uwisc.edu}
          \and
          M. C. Lam\inst{2, 3}
          }

   \institute{Department of Physics and Wisconsin IceCube Particle Astrophysics Center,
University of Wisconsin, Madison, WI 53706, USA
         \and
        School of Physics and Astronomy, Tel Aviv University, Tel Aviv 69978, Israel
         \and
        Astrophysics Research Institute, Liverpool John Moores University, IC2, LSP, 146 Brownlow Hill, Liverpool L3 5RF, UK
             }

   \date{Received \today}

% \abstract{}{}{}{}{} 
% 5 {} token are mandatory
 
  \abstract
  % context heading (optional)
  % {} leave it empty if necessary  
   {Wavelength calibration is a routine and critical part of any spectral work-flow, but many astronomers still resort to matching detected peaks and emission lines by hand.}
  % aims heading (mandatory)
   {We present RASCAL (RANSAC Assisted Spectral CALibration), a python library for automated wavelength calibration of astronomical spectrographs. RASCAL implements recent state-of-the-art methods for wavelength calibration and requires minimal input from a user}
  % methods heading (mandatory)
   {}
  % results heading (mandatory)
   {we have developed RASCAL (RANSAC-Assisted Spectral CALibration). RASCAL only requires an atlas of calibration lines, a list of peaks,
and some information about the system}
  % conclusions heading (optional), leave it empty if necessary 
   {}

   \keywords{bla --
                blabla --
                blablabla
               }

   \maketitle
%
%-------------------------------------------------------------------
\section{Introduction}
Wavelength calibration is a routine and critical part of spectral data reduction
workflow. This usually involves manually matching peaks in an arc lamp spectrum to a
catalogue of known emission lines~\citep{2010MNRAS.409.1601B}. This is an extremely
repetitive process and they accumulate to take up a substantial amount of time.
Even for an experienced user, there is a bottleneck on how fast a manual reduction
can reach.

The advances in Astronomy in the past decade has enabled multi-epoch, multi-colour
large sky-area surveys. Among the ground-base facilities, to name a few, the
Optical Gravitational Lensing Experiment~\citep[OGLE]{2015AcA....65....1U},
All-Sky Automated Survey for Supernovae~\citep[ASAS-SN]{2017PASP..129j4502K},
ZTF~\citep[Zwicky Transient Facility]{2019PASP..131a8003M},
BlackGEM~\citep[BlackGEM]{2015ASPC..496..254B}
Gravitational-wave Optical Transient Observer~\citep[GOTO]{2020SPIE11445E..7GD}
Panoramic Survey Telescope and Rapid Response System~\citep[Pan-STARRS]{2016arXiv161205560C}
and forthcoming Vera Rubin Observatory~\citep[VRO]{2019ApJ...873..111I}. Both Galactic
and extra-Galactic transient events are being found in steadily growing numbers.
However while it is relatively easy to discover, for example, a supernova, it is much
harder to collect the essential follow-up observations (both photometry and spectroscopy)
to allow classification and subsequent scientific exploitation. In addition, with the
advent of multi-messenger Astronomy with Gravitational Waves with
LIGO~\citep{2015CQGra..32g4001L} and VIRGO~\citep{2015CQGra..32b4001A} and neutrino
detectors, the IceCube~\citep{2006APh....26..155I}, rapid time-domain photometric and
spectroscopic observations are becoming even more crucial to identify and confirm their
electromagnetic counterparts. Hence, The quick turnover of the full data reductions is
important in case of Target-of-Opportunity requests, which are best performed on robotic
telescopes, and which can very quickly follow short-lived events like unusual supernovae,
cataclysmic variables, planetary deviations in microlensing events, etc. This can only be
made possible, with a global network of small/medium sized telescope dedicated for
follow-up. They, however, even when the control systems are roboticized, most of the
data curation is still depending entirely on manual operation. Despite there are various
large scale spectroscopic surveys with high volume of data output and automated data
reduction, for example, The Large Sky Area Multi-Object Fiber Spectroscopic
Telescope~\citep[LAMOST]{2012RAA....12.1197C}, WHT Enhanced Area Velocity
Explorer~\citep[WEAVE]{2012SPIE.8446E..0PD}, 4-metre Multi-Object Spectroscopic
Telescope~\citep[4MOST]{2019Msngr.175....3D}, Galactic Archaeology with
Hermes~\citep[GALAH]{2015MNRAS.449.2604D}, Radial Velocity
Experiment~\citep[RAVE]{2020AJ....160...82S}, Mapping Nearby Galaxies at Apache Point
Observatory\citep[MaNGA]{2015ApJ...798....7B} etc. They come with dedicated data
reduction pipeline, with the software pruned to optimise performance in their specific
settings. The large number of data collected simultaneously allows cross-calibration
to certain extent. These generally unportable setups make the data reduction system
difficult to be reused. The integration effort can be just as, or even more, significant
as rewriting a new dedicated software that is specific to the newly roboticized systems.
Or, in the more common case, the data reduction process is left for the individual
scientists to complete, this is both inefficient in use of highly skilled resources,
causes wasted duplication of software, and can lead to inhomogeneous data products
that are difficult to combine from different facilities and observers.

The Time Domain Astronomy section of the OPTICON recognise this series of issues and
initiated an effort to develop a portable spectrograph and a portable data reduction
software. They are now the SPRAT~\citep{2014SPIE.9147E..8HP} on the Liverpool
Telescope~\citep{2004SPIE.5489..679S} at the Observatorio Astrof{\'i}sico Roque de los
Muchachos, the MOOKODI on the Lesedi Telescope 
at the South African Astronomical Observatory, and the MISTRAL on the 1.93\,m
telescope at the Haute-Provence Observatory. On the data reduction side, a new
general spectral pipeline, ASPIRED~\citep{2020arXiv201203505L, 2020zndo...4306065L},
that is completely independent of \texttt{iraf}~\citep{1986SPIE..627..733T} is under
active development and it is close to completion. It has been a concurrent development
of the \texttt{RASCAL} software that handles the wavelength
calibration~\citep{2019arXiv191205883V, 2020zndo...4117517V}.

This following article is organised in this structure: in Section~2, we will address
the difficulties in achieving automated wavelength calibration; in Section~3,
\texttt{RASCAL} will describes in detail; the quality and the repeatability will
be demonstrated in Section~4; the deployment on various systems and some example
independent demonstrators will be reported in Section 5; in the final section, we
will conclude on the software, including the limitation, maintenance, and the potential
future development.

\section{Wavelength Calibration}
The process of wavelength calibration involves identifying distinctive emission lines
from an arc lamp spectrum, from which a polynomial can be found to map the pixel
position to the wavelength value. Automating this process may seem trivial, but in
real life application, it is not remotely easy. The strong dependency on the 
instrumental properties have made the wavelength calibration routines specific to
the instruments, to name a few, the vacuum/contamination condition of the lamp, the 
combination of elements in the lamp, the vignetting on the focal plan, the response
as function of wavelength across the detector, saturation issue. There is also likely
to be noise in the peak finding routines, for example, due to detector noise or
quantization~(e.g.\ not using sub-pixel peak finding). There may also be complications
such as blended lines - detected peaks which correspond to multiple emission lines,
such as unresolved doublets. While many observatories have published reduction
pipelines that involve wavelength calibration~\citep{2002AJ....123..485S,
2012ascl.soft03003C, 2013ApJS..208....5N}, they are not transferable to other setups.
In one specific general common pipeline, \texttt{PypeIt}~\citep{2020JOSS....5.2308P},
it uses spectral template matching for wavelength calibration, and it is done specific
per instrument. As we have discussed in the previous section, there is a growing need
for an automated solution that is easily transferable and robust to system 
re-configuration e.g. grating position, lamp type. This would be particularly
useful in sharing a single data reduction pipeline among a network of small telescope
facilities when staffing for software development and maintenance is limited.
In order to address this, we have developed RASCAL~(RANSAC-Assisted Spectral CALibration).
RASCAL only requires an atlas of calibration lines, a list of peaks, and some information
about the system. RASCAL has been developed for the ASPIRED
program~\citep{2019arXiv191205885L, 2020arXiv201203505L, 2020zndo...4306065L}
and broadly follows the algorithm presented in \citet{2018ApOpt..57.6876S}. We are
releasing RASCAL as open-source as a Python library that can be easily integrated in to
astronomical pipelines. The original paper only presents results from commercial
spectrometers, so we contribute an initial evaluation on real-world spectra
from astronomical instruments. We also present some tweaks and improvements to the
original algorithm that result in improved correspondence matching.

\subsection{Challenges}
Motivated by the computer vision algorithm, \citet{2018ApOpt..57.6876S} applies an
outline detection algorithm making use of Hough transform to quickly identify first
good guess of a pixel-to-wavelength polynomial solution. The proposition for an ideal
system is: \textit{given a set of detected peak locations in an arc spectrum ($P$ [px]),
there exists a set of matching emission lines ($A$ [$\lambda$]) for every $P$}.
Once the correspondences between $P$ and $A$ have been established, they are
used to fit a model $f(x, p) = x_{\lambda}$ where $p$ are model parameters, $x$ is a
detector location in pixels and $x_{\lambda}$ is the corresponding wavelength.
Because of the aforementioned instrumental/optical effects and imperfection in the
system, this process is not as straight forward and has to be robust against outliers.
Nevertheless, the emission lines in the atlas are assumed to be perfect and taken
from the National Institute of Standards and Technology~\citep[NIST]{NIST\_ASD}
which collates values from the literature; alternately, users can supply a line list.
No assumption is made about the peak finding routine, see next section for more details.
It is possible that some detected peaks are spurious or correspond to a line not in the
atlas. Vice versa, it is possible that some atlas lines were not detected because they
are outside the spectral range of the detector, too low in amplitude, and so on. In
fact, in the general case, the real problem we face is: \textit{for any combination of
member of $P$, there usually exists a corresponding member of $A$}. The goal is to find
the true line in $A$ for each peak in $P$. Checking all possible sets of pairs of $A$
and $P$ is computationally infeasible.Once peaks and wavelengths have been matched,
the model fitting process is largely traditional. It is important that robust fitting
methods are used, otherwise a single incorrect match can significantly skew the final
model parameters.

\section{RASCAL}
In \citet{2018ApOpt..57.6876S}, they search for \textit{plausible} sets of
correspondences which are constrained by prior knowledge about the system.
Specifically they find solutions which agree with linear approximations to the
system~($x_\lambda = Dx + c$). We likely know which lamp was used, and we also have
priors on the minimum wavelength~($c$) and dispersion ($D$) of the system. Let $A'$ be
a filtered atlas which only contains lines within a user-specified range of interest.
We allow a default tolerance of $\pm 200$\AA~to this value. We constrain $D$ based on
the number of pixels in the spectrum and the wavelength range.

The original algorithm suggests fitting models to each Hough candidate set separately
and then choosing the best. In our experience, this fails when there is a non-negligible
curvature in the model function. Instead, we consider the top $N$ candidate sets
simultaneously (we set $N = 20$ by default). For each peak, we choose the most common
best-fit atlas line from the top candidate sets. This acts somewhat like a piece-wise
linear fit and allows us to extract most of the correct matches from both the red and
blue regions of the spectrum. RANdom SAmple
Consensus~\citep[RANSAC][]{fischler_bolles_1981} is used to robustly fit
a higher order polynomial model to the candidate correspondences. This model is used
to return atlas correspondences for each peak, which can be passed to a more
sophisticated fitting function e.g. an analytical model of the instrument such as
in \citet{2013OptEn..52a3603L}.

Three built-in diagnostic plots are available, each is available in both
\texttt{plotly}~\citep{plotly} and \texttt{matplotlib}. The former can generate interactive
readable by a browser, while being able to be rendered static plots. However,
the installation can be difficult at times, so it is not our default plotting
library. In fact, it is not even installed as a dependency. On the other hand,
\texttt{matplotlib}~\citep{Hunter:2007} is the most commonly used plotting library
among \texttt{Python} users, it has no known problematic upstream dependency issue.
Therefore, it is chosen as our default plotting library. Its lack of
interactivity has a small drawback, but there are external packages that can
convert a \texttt{matplotlib} figure object into interactive plots, for example
with \texttt{mpld3}\footnote{\url{http://mpld3.github.io/}}, and this can be done completely outside the functional call
of \texttt{RASCAL}.

\subsection{Hough Transform}
Initially, every pair of peaks and emission lines are enumerated~(i.e.\ the
Cartesian product of the sets $A' \times P$.) The Hough
transform~\citep{osti_4746348} is used to search for linear correspondences
among these enumerated pairs. The result of this is a histogram of possible lines
in $(D, c)$ space. Peaks in the Hough space corresponds to straight lines
in the pixel-wavelength space which pass through (or near) many sets of
$A' \times P$, which we a candidate pair/set hereafter.

\subsection{RANSAC}
Random sample consensus, RANSAC, is an algorithm for robust fitting,
popularized by the computer vision community~\citep{fischler_bolles_1981}.

Suppose we have a dataset that contains good points~(inliers) and spurious
points~(outliers). If we randomly sample our dataset, eventually we will
pick a set of points which only contain the inliers. In that case, a model
fit to this sample should also fit the majority of the rest of our data.
Conversely if we sample an outlier by mistake, the fit will not agree with
the rest of the dataset.

This is conceptually very simple, but works extremely well in practice.
Typically the sample size is the minimum required to fit the model, so for
a linear model we would draw 2 random points. If you know the percentage of
outliers in your data it is possible to calculate some statistical estimate
of how many iterations is required before an inlier-only sample is drawn~(to
some degree of confidence). In practice, we never know the inlier-to-outlier
ratio and we just try ``a large number'' of samples\footnote{From experience,
a few hundred is ``a lot''}. This number of samples can also be adjusted as 
a hyperparameter for the algorithm to provide best results on a specific
instrument.

We also need to decide how to score a particular sample fit. In the original
version of RANSAC, the number if inliers corresponding to a fit is used. More
inliers equals a better fit. In RASCAL we use a slight modification called
M-SAC~\citep{Torr00mlesac:a} which also weights inliers by the fit error.
This is useful because it acts as a tie-breaker between two fits with the
same number of inliers.

\subsection{Procedure}
In the following article, we refer a spectrum of an arc as an \textit{arc},
the wavelength of the emission lines from an arc as \textit{lines}, whereas 
the \textit{peaks} are the pixel position where the lines are centred at,
and a collection of lines from some elements as \textit{atlas}.

The following is \textbf{not} served as an API document, we are only highlighting
some important and/or special features and limitations.

\subsubsection*{Step 1 -- Initialise a Calibrator}
A calibrator can be initialized with \texttt{Calibrator(peaks)} by passing a list of
pixel values of the the peaks identified from an arc frame. This also initialises a
logger inside to handle five levels of messages -- \textsc{debug}, \textsc{info},
\textsc{warning}, \textsc{error} and \textsc{critical}, and a \texttt{HoughTransform}
object that handles the transformation and the operations in Hough space. The
properties of the calibrator, Hough transform and the RANSAC are set with their
own functions to avoid confusion of similar name space, particularly regarding the
limits and tolerances, namely:

\subsubsection*{1. \texttt{set\_calibrator\_properties()}}
The two main parameters to set are the \texttt{num\_pix} and \texttt{pixel\_list}. The
latter can be ignored in most cases when the wavelength calibration is done with a
single detector. 

\subsubsection*{2. \texttt{set\_hough\_properties()}}


\subsubsection*{3. \texttt{set\_ransac\_properties()}}


\subsubsection*{Step 2 -- Supply atlas}
An atlas can be loaded from the built-in list of cleaned NIST list with
\texttt{load\_atlas()}; it is also possible for users to append user-defined set
of lines with \texttt{add\_user\_atlas()}.

\subsubsection*{Step 3 -- Configure the operations in Hough space}
Build the set of Hough pairs by using \texttt{set\_peaks()} which calls \texttt{\_generate\_pairs()} to
\begin{enumerate}
    \item pair up \texttt{peaks} (piexls) and \texttt{atlas} (wavelength)
    \item if \texttt{constrain\_poly} is set to \texttt{True}, \texttt{Delaunay} from \textsc{scipy} is usd to remove outlying points.
    \item order the pairs into a list of [(peak 1, atlas 1), (peak 2, atlas 2), ..., (peak n, atlas n)]
    \item Optionally, append the known (peak, atlas) pairs with \texttt{set\_known\_pairs()} which are used as anchor(s) when solving for the solutions. This function has to be used with extreme caution, a set of wrong \textbf{known pairs} will almost guarantee the best fit solution to be wrong.
\end{enumerate}

\subsubsection*{Step 4 -- Fit for the solution}
By calling the \texttt{fit()}, it calls the \texttt{\_get\_best\_model()} to do the heavy duty which starts with initializing the Hough transform instance with \texttt{HoughTransform()} to execute the \texttt{generatePhough\_points()} and then \texttt{bin\_hough\_points()} to generate (1)~\texttt{hough\_points}, and (2)~\texttt{hough\_lines}.

Run RANSAC on the Hough pairs using \texttt{\_solve\_candidate\_ransac()} to get the possible wavelength solution of all the peaks for the given \texttt{hough\_line}, which generates the \texttt{candidates} that contain the (1)~Hough solution wavelength, (2)~atlas wavelength, and (3)~weight of the solution. The most common weighted candidates will then be chosen with the \texttt{\_get\_most\_common\_candidates()} by counting and returning the \texttt{top\_n} most common Hough solution wavelength. The closely spaced candidates can be removed by setting \texttt{remove\_close} to \texttt{True}. The fit will stop and return nothing if the number of candidates is smaller than the (degree of the polynomial + 1). From the Hough pairs, build an interpolated 2D density map in Hough space. Fit the pixel-wavelength solution with polynomials. Discard the fit if the first and the second polynomials co-efficients are outside the tolerance limits. Evaluate the wavelengths with the fitted polynomials solution, compute the absolute difference between the solutions and the atlas to compute the \texttt{err} or functions. Get the \texttt{weight} function based on the intercepts and gradients form the pre-computed weight map. Combine the \texttt{err} and \texttt{weight} to get the \texttt{cost} function, if the new cost is lower, accept the new solution.


\section{Quality and Repeatability}

Automated methods vs manual/human identified peaks

Measures of fit quality (Precision/Recall, hyperparameter tuning, tests on real instruments?)

Testing framework with simulated spectra

One of the goals of RASCAL is to be agnostic to the choice of calibration lamp or target. With 
this in mind, it is not necessary to simulate a specific lamp for testing. Instead, we select
the 

\section{Dep parmloyment}
simulated lamp
\subsection{SPRAT}

\subsection{Mokoodi}

\subsection{Mistral}

\subsection{Other unofficial systems}

%--------------------------------------------------------------------
\section{Conclusions}


%--------------------------------------------------------------------
\begin{acknowledgements}
      Part of this work was supported by...
\end{acknowledgements}

% WARNING
%-------------------------------------------------------------------
% Please note that we have included the references to the file aa.dem in
% order to compile it, but we ask you to:
%
% - use BibTeX with the regular commands:
%   \bibliographystyle{aa} % style aa.bst
%   \bibliography{Yourfile} % your references Yourfile.bib
%
% - join the .bib files when you upload your source files
%-------------------------------------------------------------------

\bibliographystyle{aa}
\bibliography{rascal}

\end{document}
